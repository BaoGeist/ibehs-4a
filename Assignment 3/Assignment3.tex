\documentclass[12pt]{article}
\setlength{\oddsidemargin}{0in}
\setlength{\evensidemargin}{0in}
\setlength{\textwidth}{6.5in}
\setlength{\parindent}{0in}
\setlength{\parskip}{\baselineskip}

\usepackage{amsmath,amsfonts,amssymb,graphicx, hyperref, float}

\begin{document}

IBEHS 4A03 \hfill Assignment \#3\\
Baoze Lin, Hady Ibrahim

\hrulefill

% Custom numbering for subparts (e.g., 2.1, 2.2)
\renewcommand{\theenumii}{\arabic{enumi}.\arabic{enumii}}

\begin{enumerate}
\item Question 1
  \begin{enumerate}
    % Answer to 1.1
    \item
    The open-loop step response of the system is shown in the Simulink model (Figure \ref{fig:figure1_1}) below with the response shown in Figure \ref{fig:figure1_2}.
    
    \begin{figure}[H]
      \centering
      \includegraphics[width=0.3\textwidth]{Figures/Models/model1_1.png}
      \caption{Simulink model of the open-loop step response}
      \label{fig:figure1_1}
    \end{figure}

    \begin{figure}[H]
      \centering
      \includegraphics[width=0.5\textwidth]{Figures/figure1_1.png}
      \caption{Open-loop step response}
      \label{fig:figure1_2}
    \end{figure}

    To identify the gain and time constant for this system, we can rearrange the transfer function to isolate the gain K and time constant $\tau$:

    The given transfer function is:
    \[
    G_p(s) = \frac{6}{s + 2}
    \]

    To express this in the standard first-order form, which is:
    \[
    G(s) = \frac{K}{\tau s + 1}
    \]
    where \( K \) is the gain and \( \tau \) is the time constant.

    We rearrange \( G_p(s) \) as follows:
    \[
    G_p(s) = \frac{6}{2(\frac{1}{2}s + 1)} = \frac{3}{\frac{1}{2}s + 1}
    \]

    Thus, we identify the gain \( K \) and the time constant \( \tau \) as:
    \[
    K = 3, \quad \tau = \frac{1}{2}
    \]

    % Answer to 1.2
    \item 
    The closed-loop feedback control with a proportional controller is shown below in Figure \ref{fig:figure1_3}. The transfer function $G_c(s)$ can also be represented by a proportional gain $K_c$.

    \begin{figure}[H]
      \centering
      \includegraphics[width=0.7\textwidth]{Figures/Models/model1_2.png}
      \caption{Simulink model of the closed-loop feedback control with a proportional controller}
      \label{fig:figure1_3}
    \end{figure}
    
    \pagebreak
    
    % Answer to 1.3
    \item 
    The Simulink model is shown below in Figure \ref{fig:figure1_4}. The response of the system is shown in Figure \ref{fig:figure1_5}.

    \begin{figure}[H]
      \centering
      \includegraphics[width=0.5\textwidth]{Figures/Models/model1_3.png}
      \caption{Simulink model of the closed-loop feedback control with a proportional controller}
      \label{fig:figure1_4}
    \end{figure}

    \begin{figure}[H]
      \centering
      \includegraphics[width=0.8\textwidth]{Figures/figure1_3.png}
      \caption{Closed-loop step response}
      \label{fig:figure1_5}
    \end{figure}

    % Answer to 1.4
    \item
    The closed-loop transfer function \( T(s) \) is given by:
    \[
    T(s) = \frac{Y'(s)}{R'(s)} = \frac{G_c(s)G_p(s)}{1 + G_c(s)G_p(s)}
    \]

    Where \( G_c(s) = K_c = 4 \) (proportional controller) and \( G_p(s) = \frac{6}{s+2} \) (plant transfer function). Substituting the transfer functions:
    \[
    T(s) = \frac{4 \cdot \frac{6}{s+2}}{1 + 4 \cdot \frac{6}{s+2}}
    \]

    Simplifying:
    \[
    T(s) = \frac{\frac{24}{s+2}}{1 + \frac{24}{s+2}} = \frac{24}{s+2+24} = \frac{24}{s+26}
    \]

    Thus, the closed-loop transfer function is:
    \[
    T(s) = \frac{24}{s+26}
    \]

    We then apply the Final Value Theorem, which states that:
    \[
    \lim_{t \to \infty} y(t) = \lim_{s \to 0} sY(s)
    \]

    For a unit step input \( R'(s) = \frac{1}{s} \), the Laplace transform of the output is:
    \[
    Y'(s) = T(s)R'(s) = \frac{24}{s+26} \cdot \frac{1}{s}
    \]

    Applying the FVT:
    \[
    \lim_{t \to \infty} y(t) = \lim_{s \to 0} s \cdot \frac{24}{(s+26)s} = \lim_{s \to 0} \frac{24}{s+26} = \frac{24}{26} = \frac{12}{13} \approx 0.923
    \]

    The steady-state value of the output is approximately \( 0.923 \). Since the system is operating on displacement from a rock assembly with a step input, the set point is 1. Thus, the offset of this system is approximately \( 0.923 \). Referring to the step response in Figure \ref{fig:figure1_5}, we can see the offset is approximately \( 1 - 0.923 = 0.077\).

    \pagebreak

    % Answer to 1.5
    \item
    The closed-loop system for controller gains $K_c = {0.25, 1, 2, 4, 10}$ are shown in Figure \ref{fig:figure1_6}. Observing the results, we can see that the offset of the system decreases as the proportional gain increases.

    \begin{figure}[H]
      \centering
      \includegraphics[width=0.8\textwidth]{Figures/figure1_5.png}
      \caption{Closed-loop step response for different controller gains}
      \label{fig:figure1_6}
    \end{figure}

    % Answer to 1.6
    \item
    We want to find the controller gain \(K_c\) such that the offset is less than 0.08. The offset is given by:

    \[
    \text{Offset} = \lim_{t \to \infty} (r(t) - y(t))
    \]

    Using the Final Value Theorem:

    \[
    \text{Offset} = \lim_{s \to 0} \left(1 - \frac{K_c \cdot G_p}{1 + K_c \cdot G_p}\right)
    \]

    With \(G_p = \frac{6}{s+2}\), we evaluate at \(s=0\):

    \[
    \text{Offset} = 1 - \frac{K_c \cdot \frac{6}{0+2}}{1 + K_c \cdot \frac{6}{0+2}} = 1 - \frac{3K_c}{1 + 3K_c}
    \]

    We require that the offset be less than 0.08:

    \[
    0.08 > 1 - \frac{3K_c}{1 + 3K_c}
    \]

    Rearranging the inequality:

    \[
    \frac{3K_c}{1 + 3K_c} > 1 - 0.08 = 0.92
    \]

    \[
    3K_c > 0.92(1 + 3K_c)
    \]

    \[
    K_c > 3.8333...
    \]

    Therefore, the controller gain \(K_c\) must be greater than approximately 3.83 to achieve an offset less than 0.08.

    % Answer to 1.7
    \item
    We change from the proportional controller to the proportional-integral controller by adjusting the transfer function \(G_c(s)\) to include an integral term. The transfer function for the PI controller is given by:

    \[
    G_c(s) = K_c \left(1 + \frac{1}{\tau_I s}\right)
    \]

    Where \(K_c = 3\) and \(\tau_I = 0.05\).  The response of the system is shown in Figure \ref{fig:figure1_7}.

    \begin{figure}[H]
      \centering
      \includegraphics[width=0.8\textwidth]{Figures/figure1_7.png}
      \caption{Closed-loop step response with a PI controller}
      \label{fig:figure1_7}
    \end{figure}

    % Answer to 1.8
    \item
    Given the below two transfer functions in our system:

    \[
    G_c = K_c \left( 1 + \frac{1}{\tau_I s} \right)
    \]

    \[
    G_p = \frac{6}{s+2} = \frac{K_p}{\tau_p s + 1}
    \]

    where,

    \[
    K_p = 3, \quad \tau_p = \frac{1}{2}, \quad K_c = 3, \quad \tau_I = 0.05
    \]

    Together, the closed-loop transfer function for the system becomes:

    \[
    \frac{Y'(s)}{R'(s)} = \frac{\tau_I s + 1}{\frac{\tau_I \tau_p}{K_c K_p} s^2 + \tau_I \frac{(1 + K_c K_p)}{K_c K_p} s + 1}
    \]

    We can use the equations as presented in the lecture to calculate the time constant \( \tau \) and the damping factor \( \zeta \).

    \[
    \tau = \sqrt{\frac{\tau_I \tau_p}{K_c K_p}} = \sqrt{\frac{(0.05)(\frac{1}{2})}{(3)(3)}}
    = \sqrt{0.00278}
    \]

    \[
    \tau \approx 0.0527
    \]

    \[
    \zeta = \frac{1}{2} (1 + K_c K_p) \sqrt{\frac{\tau_I}{K_c K_p \tau_p}}
 = \frac{1}{2} (1 + 9) \sqrt{\frac{0.05}{(3)(3)(\frac{1}{2})}}
    \]

    \[
    \zeta \approx 0.526
    \]

    We also derive the closed-loop gain \( K \).

    \[
    K = \tau_I s + 1 = 0.05s + 1
    \]

    The calculated time constant is approximately \( \tau \approx 0.0527 \), which suggests the system should settle relatively quickly. The expected settling time, approximately \( 4\tau = 0.21s \), can be seen in Figure \ref{fig:figure1_7}, where it settles around that point. \linebreak

    The damping factor was found to be \( \zeta \approx 0.526 \), suggesting a slightly underdamped response. This means the system should exhibit small oscillations but should not overshoot excessively, as observed in the above figure. \linebreak

    The closed-loop gain is given by \( K = 0.05s + 1 \), which means the gain varies with frequency rather than being a simple constant. Evaluating it at steady-state (\( s = 0 \)), we get \( K(0) = 1 \), which confirms that the system should track a step input without steady-state error. Over time, our system approaches 1 as expected.

    % Answer to 1.9
    \item
    The closed-loop system for integral times \( \tau_I = {0.01, 0.05, 0.1, 0.5, 1, 2} \) are shown in Figure \ref{fig:figure1_8}. As the integral time decreases, the system has an increased speed of response but also a higher relative overshoot.

    \begin{figure}[H]
      \centering
      \includegraphics[width=0.8\textwidth]{Figures/figure1_9.png}
      \caption{Closed-loop step response for different integral times}
      \label{fig:figure1_8}
    \end{figure}

    From comparing the different integral time constants, we can see that \(\tau_I\) dictates in a PI controller the trade-off between response speed and stability. Small values (e.g., \(\tau_I = 0.01\)) yield fast responses but can cause excessive overshoot and oscillations. Larger values (e.g., \(\tau_I = 2\)) improve stability with less overshoot but result in slower responses. Tuning \(\tau_I\) aims to find an optimal value, like \(\tau_I = 0.4\), that balances speed and stability according to the specific process requirements. This is due to the mathematical relationship between \(\tau_I\) and the damping factor and time constant, which affect the system's response characteristics.

    % Answer to 1.10
    \item
    The surface plots are shown below in Figure \ref{fig:figure1_9} and Figure \ref{fig:figure1_10}. The surface plots show the relationship between the integral time constant \( \tau_I \) and the damping factor \( \zeta \) with the settling time and the relative overshoot.

    \begin{figure}[H]
      \centering
      \includegraphics[width=0.6\textwidth]{Figures/figure1_10a.png}
      \caption{Surface plot of settling time and integral time constant}
      \label{fig:figure1_9}
    \end{figure}

    \begin{figure}[H]
      \centering
      \includegraphics[width=0.6\textwidth]{Figures/figure1_10b.png}
      \caption{Surface plot of relative overshoot and integral time constant}
      \label{fig:figure1_10}
    \end{figure}

    The surface plots for the closed-loop time constant (\(\tau\)) and damping factor (\(\zeta\)) reveal important insights into the system's dynamics and the trade-offs between speed and stability. The time constant plot shows that \(\tau\) decreases as the proportional gain (\(K_c\)) increases, indicating a faster system response. However, very high \(K_c\) values can lead to instability, as seen in the damping factor plot. Similarly, smaller integral time constants (\(\tau_I\)) reduce \(\tau\), but overly small \(\tau_I\) can destabilize the system. The goal is to minimize \(\tau\) for a fast response while ensuring the system remains stable.

    The damping factor plot highlights that a value of \(\zeta\) close to 1 is ideal, as it represents critical damping, where the system responds quickly without oscillations. For small \(K_c\), the system is underdamped (\(\zeta < 1\)), leading to oscillations, while very large \(K_c\) results in overdamping (\(\zeta > 1\)), causing a sluggish response. Larger \(\tau_I\) values increase \(\zeta\), while smaller \(\tau_I\) reduce it, potentially leading to instability.

    To optimize the system, we aim to balance speed and stability by selecting \(K_c\) and \(\tau_I\) values that minimize \(\tau\) while keeping \(\zeta\) close to 1. From the plots, moderate \(K_c\) values (e.g., 2–4) and small-to-moderate \(\tau_I\) values (e.g., 0.5–1) are likely to achieve this balance, providing a fast and stable response.

    % Answer to 1.11
    \item
    The curve that represents the intersection between the surface $\zeta = f(K_c, \tau_I)$ and plane $\zeta = 1$. The curve is shown in Figure \ref{fig:figure1_11}.

    \begin{figure}[H]
      \centering
      \includegraphics[width=0.7\textwidth]{Figures/figure1_11.png}
      \caption{Intersection curve between the surface and plane}
      \label{fig:figure1_11}
    \end{figure}

    % Answer to 1.12
    \item
    To optimize the system, we want to choose parameters that also have the lowest closed-loop time constant, as well as having a damping factor of 1. The closed-loop time constant is thus overlayed over the Figure \ref{fig:figure1_11} in Figure \ref{fig:figure1_12}.

    \begin{figure}[H]
      \centering
      \includegraphics[width=0.7\textwidth]{Figures/figure1_12.png}
      \caption{Intersection curve between the surface and plane with the closed-loop time constant}
      \label{fig:figure1_12}
    \end{figure}

    Given the constrainst that $K_c \leq 5$, the optimal values that follows the rational above occurs in Figure \ref{fig:figure1_12} at $K_c = 5$ and $\tau_I = 0.117187$. This point represents the optimal balance between speed and stability, ensuring a fast response without oscillations.

    % Answer to 1.13
    \item
    Graphing the above optimal tuning parameters, we get the following results for y\(t\). It exhibits the system characteristics that we were optimizing for in thh previous questions -- we have the quickest response that could occur while critically damped, as shown in Figure \ref{fig:figure1_13}.

    \begin{figure}[H]
      \centering
      \includegraphics[width=0.7\textwidth]{Figures/figure1_13.png}
      \caption{System response with optimal tuning parameters}
      \label{fig:figure1_13}
    \end{figure}

  \end{enumerate}

  

\pagebreak

\item Question 2
  \begin{enumerate}
    \item placeholder
  \end{enumerate}

\end{enumerate}

\end{document}
