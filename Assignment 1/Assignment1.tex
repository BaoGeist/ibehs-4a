\documentclass[12pt]{article}
\setlength{\oddsidemargin}{0in}
\setlength{\evensidemargin}{0in}
\setlength{\textwidth}{6.5in}
\setlength{\parindent}{0in}
\setlength{\parskip}{\baselineskip}

\usepackage{amsmath,amsfonts,amssymb,graphicx, hyperref, float}

\title{Your Project Title}

\begin{document}

IBEHS 4A03 \hfill Assignment \#1\\
Baoze Lin, Hady Ibrahim

\hrulefill

% Custom numbering for subparts (e.g., 2.1, 2.2)
\renewcommand{\theenumii}{\arabic{enumi}.\arabic{enumii}}

\begin{enumerate}
\item Question 1
  \begin{enumerate}
  % ANSWER TO 1.1
  \item yapyapyap

  \end{enumerate}
\newpage

\item Question 2
  \begin{enumerate}
  % ANSWER TO 2.1
  \item 

  The ordinary differential equation (ODE) governing the temperature \( T(t) \) is:

  \[
  \frac{dT(t)}{dt} = \frac{Q_f(t) - UA(T(t) - T_a)}{\rho V c_p}
  \]

  For this problem, the furnace is off, so \( Q_f(t) = 0 \). Substituting this into the equation:

  \[
  \frac{dT(t)}{dt} = \frac{-UA(T(t) - T_a)}{\rho V c_p}
  \]

  At steady state, the temperature \( T(t) \) no longer changes with time, so:

  \[
  \frac{dT(t)}{dt} = 0
  \]

  Substitute this condition into the ODE:

  \[
  0 = \frac{-UA(T(t) - T_a)}{\rho V c_p}
  \]

  Solve:

  \[
  T(t) - T_a = 0
  \]

  Thus:

  \[
  T(t) = T_a
  \]

  % ANSWER TO 2.2
  \item
    Using the derived equation for \( T(t) \), the behaviour of the furnace as defined below, and the given definition of one step of the univariate Euler's method, we can plot Figure 1.

  \[
  Q_f(t) =
  \begin{cases} 
  0 & \text{When } T(t) > 23^\circ \text{C}, \\
  1.5 \times 10^6 & \text{When } T(t) < 17^\circ \text{C}, \\
  \text{unchanged} & \text{For all } 17 \leq T(t) \leq 23^\circ \text{C}.
  \end{cases}
  \]

  \begin{figure}[H]
    \centering
    \includegraphics[width=\textwidth]{Figures/figure22.png}
    \caption{Temperature and Furnace Input Over Time}
    \label{fig:figure22} 
  \end{figure}

  % ANSWER TO 2.3
  \item 
  To calculate the number of standard cubic meters of natural gas consumed, we follow the below equation:

  Define $t_{on} \triangleq$  time the furnace was on in hours. The Matlab code says it was 1.182 hours\\
  Define $Q_{f,on} \triangleq$ as the furnace heat input. \\
  Define $\rho_e \triangleq$ as the energy density. \\
  Define $e \triangleq$ as the efficiency. \\

  \[
  \text{Volume of gas consumed} = \frac{t_{on} Q_{f,\text{on}}}{\rho_e e} = \frac{(1.182)(1.5 \times 10^6)}{(9 \times 10^6)(0.9)} = 0.2188888
  \]
  \[
  \approx 0.219 \, \text{standard cubic meters}
  \]

  We also confirm with unit analysis that we get cubic meters.

  \[
  \text{Volume of gas consumed} = \frac{t_{\text{on}} Q_{f,\text{on}}}{\rho_e e} = \frac{\text{hour} \cdot \frac{\text{cal}}{\text{hour}}}{\frac{\text{cal}}{\text{m}^3}} = \text{m}^3
  \]

  % ANSWER TO 2.4
  \item
  With a new oscillating defintion of Ta, we get Figure 2 below demonstrating the temperature and furnace input over time.

  \begin{figure}[H]
    \centering
    \includegraphics[width=\textwidth]{Figures/figure24.png}
    \caption{Temperature and Furnace Input Over Time}
    \label{fig:figure24}
  \end{figure}

  The simulation results align with the expected behavior of the system. The room temperature \( T(t) \) is maintained within the desired range of \( 17^\circ \text{C} \) to \( 23^\circ \text{C} \) as the furnace responds to changes in the ambient temperature \( T_a(t) \). When \( T_a(t) \) is lower, the furnace operates more frequently to offset increased heat loss, while at higher \( T_a(t) \), the furnace operates less often due to reduced heat loss. This behavior reflects the thermal dynamics of the system and the influence of the bang-bang control logic.


  \end{enumerate}

  \pagebreak

\item Question 3
  \begin{enumerate}
  % ANSWER TO 3.1
  \item 
    We are given the following ODE describing the current going thorugh the system.

    \[
    V_s(t) = Ri(t) + L\frac{di(t)}{dt}
    \]

    During steady state, the current \( i(t) \) is constant, so the derivative of \( i(t) \) with respect to time is zero. Substituting this into the ODE:

    \[
    V_s(t) = Ri(t) + L(0)
    \]

    Rearranging, we get the following equation for $i(t)$:

    \[
    i(t) = \frac{V_s(t)}{R}
    \]

    This is useful when designing a cirtcuit with a target current i(t) because you can control the voltage source to achieve the desired current while the resistance will always be a constant property of the circuit's hardware. 

  % ANSWER TO 3.2
  \item 
    I don't remember how to use ode45, so I referenced [1] to get started. The results of the Matlab code was plotted below in Figure 3.

    \begin{figure}[H]
      \centering
      \includegraphics[width=\textwidth]{Figures/figure32.png}
      \caption{Current Over Time}
      \label{fig:figure32}
    \end{figure}

  % ANSWER TO 3.3
  \item 
    
    \[
    V_s(t) = Ri(t) + L\frac{di(t)}{dt}
    \]
  
    Based on the ODE above, inductance $L$ should have no effect on the steady state current $i(t)$ because the derivative of $i(t)$ with respect to time is zero, and inductance is a coeffecient on the derivative. This means that the inductance $L$ term in the ODE will not affect the current $i(t)$ when the system reaches steady state, and the current $i(t)$ will be determined solely by the voltage source $V_s(t)$ and the resistance $R$ in the circuit. This can be seen below in Figure 4, where changing R causes the steady state current $i(t)$ to change, while changing L has no effect on the steady state current $i(t)$.

    \begin{figure}[H]
      \centering
      \includegraphics[width=\textwidth]{Figures/figure33.png}
      \caption{Current Over Time}
      \label{fig:figure33}
    \end{figure}
  
    % ANSWER TO 3.4
    \item
      Figure 5 and 6 below demonstrates the LTI nature of the system. An LTI system is defined as a system that is both Linear and Time Invariant. A linear system is one that follows the condition that if $x(t) = \alpha x_1(t) + \beta x_2(t), \text{ then } y(t) = \alpha y_1(t) + \beta y_2(t)$. That is, the output of a system is a linear combination of the inputs. We can observe in Figure 5 that the combination of the inputs $V_s1 = 5V$ and $V_s2 = 10V$ results in the outputs $i_1(t)$ and $i_2(t)$, which is equal to $i_3(t)$.$i_3(t)$ is the resultant output from the combination input of $V_s3 = V_s1 + V_s2 = 15V$. This is exhibits the linear property. We can also observe the system is time-invariant by looking at the equation --- the equation has no coeffecients that change over time. We can also see in Figure 6 that when taking the input and adding time delay, it is equivalent to taking the original output and time shifting that. This proves the equation is time-invariant.

    \begin{figure}[H]
      \centering
      \includegraphics[width=\textwidth]{Figures/figure34.png}
      \caption{Comparing Linear Combination of Inputs to an Equivalent Output Over Time}
      \label{fig:figure34}
    \end{figure}

    \begin{figure}[H]
      \centering
      \includegraphics[width=\textwidth]{Figures/figure35.png}
      \caption{Comparing a Time-Shifted Input with the Time-Shifted Output of the Original Signal}
      \label{fig:figure35}
    \end{figure}

  \end{enumerate}
\newpage

\item Question 4
  \begin{enumerate}
  \item Part A  % This will appear as "4.1"

  SOLUTION

  \item Part B  % This will appear as "4.2"

  SOLUTION
  \end{enumerate}
\newpage

\item Question 5
  \begin{enumerate}
  \item Part A  % This will appear as "5.1"

  SOLUTION

  \item Part B  % This will appear as "5.2"

  SOLUTION
  \end{enumerate}
\newpage

\end{enumerate}

\pagebreak

\begin{thebibliography}{9}
  \bibitem{matlab_ode45}
  MATLAB Documentation: \texttt{ode45}. Available at: 
  \url{https://www.mathworks.com/help/matlab/ref/ode45.html}.
\end{thebibliography}

\end{document}
